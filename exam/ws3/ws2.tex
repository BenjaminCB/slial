
\documentclass{article}


\usepackage{amsmath}
\usepackage{amssymb}
\usepackage{graphicx}

\graphicspath{ {./images/} }

\setlength{\parskip}{0.5\baselineskip plus2pt minus2pt}
\setlength\parindent{0pt}
\setlength{\textheight}{9in}

\title{WS3 Exam presentation}

\begin{document}
\section{Delopgave 2}
Betragt en vektor $[a,b] \in \mathbb{R}^2$ forskellig fra nulvektoren. Givens rotation er en ortogonal transformation, som afbilleder $[a,b]$ til en vektor $[d,0] \in \mathbb{R}^2$.

\subsection{}
Brug s\ae tning 7 i afsnit 6.2 [Lay], til at bestemme $[d]$ givet $[a,b]$

Let $U$ be an $m$ x $n$ matrix with orthonormal columns, and let $x, y \in \mathbb{R}^n$. Then.

\begin{enumerate}
    \item $||Ux|| = ||x||$
    \item $(Ux) \cdot (Uy) = x \cdot y$
    \item $(Ux) \cdot (Uy) = 0 \text{ if and only if } x \cdot y = 0$
\end{enumerate}

\begin{equation}
    \sqrt{a^2 + b^2} = \sqrt{d^2 + 0^2} = \sqrt{d^2} = d
\end{equation}

\subsection{}
Tjek at
\begin{equation}
    G =
    \begin{bmatrix}
        c & s \\
        -s & c
    \end{bmatrix}
\end{equation}
Hvor $c = a/\sqrt{a^2 + b^2}$ og $b/\sqrt{ a^2 + b^2 }$ er en ortogonal matrix som afbilleder $[a,b]$ til $[d,0]$.

\begin{equation}
    c \cdot s + (-s) \cdot c = 0
\end{equation}

\begin{equation}
    \begin{array}{c}
        \sqrt{c^2 + (-s)^2} \\\\
        \sqrt{c^2 + s^2} \\\\
        \sqrt{(\frac{a}{\sqrt{a^2+b^2}})^2 + (\frac{b}{\sqrt{a^2+b^2}})^2} \\\\
        \sqrt{\frac{a^2}{a^2+b^2} + \frac{b^2}{a^2+b^2}} \\\\
        \sqrt{\frac{a^2 + b^2}{a^2 + b^2}} \\\\
        1
    \end{array}
\end{equation}

\begin{equation}
    \begin{bmatrix}
        c & s \\
        -s & c
    \end{bmatrix}
    \cdot
    \begin{bmatrix}
        a \\
        b
    \end{bmatrix}
    =
    \begin{bmatrix}
        a \cdot c \\
        a \cdot (-s)
    \end{bmatrix}
    +
    \begin{bmatrix}
        b \cdot s \\
        b \cdot c
    \end{bmatrix}
    =
    \begin{bmatrix}
        a \cdot c + b \cdot s \\
        a \cdot (-s) + b \cdot c
    \end{bmatrix}
\end{equation}

Focus on bottom and insert values
\begin{equation}
    a \cdot \frac{-b}{\sqrt{ a^2 + b^2 }} + b \cdot \frac{a}{\sqrt{a^2 + b^2}}
    =
    -\frac{ab}{\sqrt{ a^2 + b^2 }} + \frac{ab}{\sqrt{a^2 + b^2}}
    =
    0
\end{equation}

Focus and top and insert values
\begin{equation}
    \begin{array}{c}
    a \cdot \frac{a}{\sqrt{ a^2 + b^2 }} + b \cdot \frac{b}{\sqrt{a^2 + b^2}} \\\\
    \frac{a^2}{\sqrt{ a^2 + b^2 }} + \frac{b^2}{\sqrt{a^2 + b^2}} \\\\
    \frac{a^2 + b^2}{\sqrt{ a^2 + b^2 }} \\\\
    \frac{a^2 + b^2 }{\sqrt{a^2 + b^2}} \cdot \frac{\sqrt{a^2 + b^2}}{\sqrt{a^2 + b^2}} \\\\
    \frac{(a^2 + b^2) \cdot \sqrt{a^2 + b^2}}{a^2 + b^2} \\\\
    \sqrt{a^2 + b^2}
    \end{array}
\end{equation}

\subsection{}
Lad os nu betragte en vektor $x \in \mathbb{R}^m, m\geq 2$, s\aa ledes at $x_i=a$ og $x_j = b$, $i < j$. Ve beregner $c$ og $s$ som i det sidste delsp\o rsm\aa l, og lad $G(i,j,a,b)$ v\ae re en $m$ x $m$ matrix, med alle r\ae kke/s\o jle som i identitetsmatrix, bortset fra r\ae kke/s\o jle i og j, hvor vi "inds\ae tter" matrix G.

\begin{equation}
    G(i,j,a,b) =
    \begin{bmatrix}
        1 & 0 & 0 & 0 & 0 \\
        0 & c & 0 & s & 0 \\
        0 & 0 & 1 & 0 & 0 \\
        0 & -s& 0 & c & 0 \\
        0 & 0 & 0 & 0 & 1
    \end{bmatrix}
\end{equation}

Tjek at $G(i,j,a,b)$ er en ortogonal matrix og at

\begin{equation}
    G(i,j,a,b)x =
    \begin{bmatrix}
        x_1 \\
        \vdots \\
        x_{i-1} \\
        d \\
        x_{i+1} \\
        \vdots \\
        x_{j-1} \\
        0 \\
        x_{j+1} \\
        \vdots \\
        x_m
    \end{bmatrix}
\end{equation}

Hvis man tager udgangspunkt i identitetsmatrix som vi ved er ortogonal, er tre scenarier der skal kigges paa

\begin{enumerate}
    \item To u\ae ndrede s\o jler
    \item En u\ae ndredet s\o jle og en \ae ndret
    \item To \ae ndrede s\o jler
\end{enumerate}

\begin{equation}
    0 \cdot 0 + \dots + c \cdot s + \dots + (-s) \cdot c + \dots + 0 \cdot 0 = 0
\end{equation}

\begin{equation}
    \begin{array}{c}
        x_1 \cdot 0 + \dots + x_i \cdot c + \dots + x_j \cdot s + \dots + x_m \cdot 0 \\
        x_i \cdot c + x_j \cdot s \\
        a \cdot c + b \cdot s \\
        d
    \end{array}
\end{equation}

\begin{equation}
    x_i \cdot (-s) + x_j \cdot c = a \cdot (-s) + b \cdot c = 0
\end{equation}

\subsection{}
Forklar, hvorfor produktet af ortogonal matricer er en ortogonal matrix. Dvs, hvis $Q_1,Q_2, \dots ,Q_k$ er ortogonale matricer, forklar hvorfor matricen $Q_1Q_2 \dots Q_k$ er ortogonal.

To be orthogonal means to satisfy

\begin{equation}
    A^T = A^{-1} \implies A^TA=AA^T=I
\end{equation}

\begin{equation}
    (M \cdot N)^T \cdot (M \cdot N) = N^T \cdot M^T \cdot M \cdot N = N^T \cdot N = I
\end{equation}

\begin{equation}
    (M \cdot N) \cdot (M \cdot N)^T = M \cdot N \cdot N^T \cdot M^T = M \cdot M^T = I
\end{equation}

\end{document}
